\documentclass[12pts]{article}
\usepackage[utf8x]{inputenc}
\usepackage[spanish, es-tabla]{babel}
\usepackage{graphicx}
\usepackage{float}
\usepackage{verbatim}
\usepackage{multicol}
\usepackage{amsmath}

\title{Estudio de la Ecuación de Lattice-Boltzman Cuántica}
\author{Juan B. Benavides y Juan P. Vanegas}
\date{\today}

\begin{document}
\maketitle
\section{Introducción}\label{sec:intro}
Desde el inicio de la mecanica cuantica la gente se dio cuenta que analizar sistemas era muy 
dificil para cualquier cosa que no fuera algo sencillo. Las herramientas computacionales 
permiten analizar sistemas más complejos, aumentando nuestro conocimiento de la mecanica cuantica.
\subsection{Autómata Celular}
Un autómata celular es básicamente una matriz y un conjunto de reglas de evolución.

\subsection{Lattice Boltzmann}
A partir de la dinámica de gases es posible derivar una relación entre las variables macroscopicas
de un fluido y una función dada en el punto, $f_i(\textbf{x}, t)$ donde

\begin{equation*}
    \rho(\textbf{x}, t) = \sum_i f_i(\textbf{x}, t)
\end{equation*}

\begin{equation*}
    \rho\textbf{u}(\textbf{x}, t) = \sum_i \textbf{c}_i f_i(\textbf{x}, t)
\end{equation*}

Discretizando la ecuación de Lattice Boltzmann para el espacio y el tiempo tenemos

\begin{equation*}
    f_i(\textbf{x}+\textbf{c}_i \Delta t, t+\Delta t) = f_i(\textbf{x}, t) + \Omega_i(\textbf{x}, t)
\end{equation*}

Donde el operador de colisión $\Omega_i$ más simple es el de Bhatnagar-Gross-Krook (BGK), 
definido como 

\begin{equation*}
    \Omega_i = \frac{f_i-f_i^{eq}}{\tau}\Delta t
\end{equation*}

Donde la función de equilibrio depende del sistema que se quiera modelar (fluidos, difusión, etc.).


\section{Metodos para simular MC}

\subsection{Meyer}
\subsubsection{Automata Celular}
 Para nuestro 
caso tomamos una matriz unidimensional, y usando la ecuación de Schrödinger encontramos la 
regla de evolución 
\begin{equation}\label{eq:evolution}
    \phi_{t+1} = matriz*\phi_t
\end{equation}
\subsubsection{Lattice Gas}
Extendiendo nuestro sistema podemos hallar una analogía con un automata celular de difusión

\subsection{Succi}
 Para nuestro caso, la función de equilibrio es
\begin{equation*}
	f_i^{eq} = nofuckingidea
\end{equation*}

\section{Simulación}
\section{Resultados}
\section{Conclusión}


\end{document}