\documentclass[12pts, letterpaper]{article}
\usepackage[utf8]{inputenc}
\usepackage[spanish, es-tabla]{babel}
\usepackage{csquotes} %babel asked for this
\usepackage{graphicx}
\usepackage{float}
\usepackage{verbatim}
\usepackage{multicol}
\usepackage{amsmath}

\usepackage{biblatex}
\addbibresource{ref.bib}

\title{Simulación de Dinámica Cuántica utilizando Lattice Boltzmann}
\author{Juan B. Benavides y Juan P. Vanegas}
\date{Febrero de 2020}

\begin{document}
\maketitle
\section{Introducción}
La mecanica cuántica es una de las ramas de la física más relevantes hoy en día, con un 
número importante de físicos trabajando en ella. Gracias a ella es posible 
comprender el funcionamiento de la naturaleza a escalas pequeñas, y con eso, estudiar 
sistemas cruciales para nosotros, desde cómo funciona el interior de las estrellas, hasta 
el comportamiento de proteínas esenciales para la vida. Una diferencia crucial entre la 
descripción clásica del mundo y la cuántica es la falta de determinismo, que obliga a 
realizar un estudio basado en estadísticas y probabilidades. Es por esto que encontramos 
elementos como la función de onda, que en sí misma no representa ninguna realidad física, 
pero al realizar diferentes operaciones sobre ella podemos hallar los valores esperados 
para diferentes observables.
\\ \\
Debido a la complejidad matemática que trae consigo esta descripción de la realidad, pocos 
sistemas tienen una solución analítica. Dentro del gran conjunto de sistemas no resueltos 
a escalas atómicas podría yacer la clave para un entendimiento más profundo de la realidad,
por lo que su estudio resulta fundamental. El estudio de estos sistemas utilizando 
herramientas computacionales introduce una visión totalmente nueva sobre estos, abriendo 
todo un nuevo paradigma de estudio. Por un lado, permite ver la evolución de estos 
sistemas más complejos, sin necesidad de solucionar analíticamente las ecuaciones que los 
describen, permitiendo avanzar en investigaciones claves para el desarrollo de la ciencia. 
Por otro lado, es posible implementar nuevos algoritmos que incluyan correción de efectos 
cuánticos, o incluso, desarrollar algoritmos para computación cuántica.
\\ \\
Dentro de los métodos computacionales para simular sistemas físicos, podemos destacar dos 
que a primera vista pueden ser apropiados para la simulación de mécanica cuántica: 
Autómatas Celulares y Lattice Boltzmann. El motivo por el cuál destacan se debe a la 
naturaleza de su estructura: No se basan en comportamientos deterministas de particulas, 
sino en la evolución de un sistema en la que cada punto del espacio interactúa de acuerdo 
con alguna regla con sus vecinos, similar a la evolución de los sistemas cuánticos.

\subsection{Autómata Celular}
El método de autómatas celulares, desarrollado en 1940 por Stanislaw Ulam y John von 
Neumann, consta en un arreglo de celdas y un conjunto de reglas de evolución, que 
relacionan el comportamiento de cada celda basado en sus vecinas. Este método resulta 
especialmente útil para modelar sistemas difusivos, biofísicos, químicos o incluso 
sistemas computacionales en sí mismo. Dentro de los modelos de autómatas celulares más 
destacados vale la pena mencionar el juego de la vida de John Conway.

\subsection{Lattice Boltzmann}
El método de Lattice Boltzmann nace de la dinámica de fluidos y de la ecuación de 
Boltzmann, que, al ser discretizada, permite obtener una ecuación que depende 
exclusivamente del comportamiento local del sistema, facilitando su implementación 
computacional en serie o en paralelo, haciéndolo un método muy eficiente en cuanto a 
tiempo de cómputo. Para esto partimos primero de una función de parámetros discretos $f$, 
que se relaciona con las variables macroscópicas del sistema mediante las relaciones 

\begin{equation*}
    \rho(\textbf{x}, t) = \sum_i f_i(\textbf{x}, t)
\end{equation*}

\begin{equation*}
    \rho\vec{u}(\textbf{x}, t) = \sum_i \vec{v_i} f_i(\textbf{x}, t)
\end{equation*}

Donde $\rho$ y $\vec{u}$ son la densidad de masa y el campo de velocidades del fluido (o 
su equivalente dependiendo del sistema que se simule). Por otro lado, el subíndice $i$ 
pertenece a un conjunto de direcciones discreto, en las cuales se propaga la información.
Al discretizar la ecuación de Boltzmann encontramos que esta función obedece la ecuación

\begin{equation}\label{eq:LB}
    f_i(\textbf{x}+\vec{c_i} \Delta t, t+\Delta t) = f_i(\textbf{x}, t) + \Omega_i(\textbf{x}, t)
\end{equation}

En donde se ve la localidad del sistema, dependiendo únicamente de los puntos vecinos y de 
un operador colisión $\Omega$. Utilizando la aproximación de Bhatnagar-Gross-Krook (BGK), 
tenemos que el operador colisión está dado por 

\begin{equation*}
    \Omega_i = -\frac{f_i-f_i^{eq}}{\tau}\Delta t
\end{equation*}

Donde la función de equilibrio depende del sistema que se quiera modelar, y el tiempo de 
relajación $\tau$ generalmente se toma como $\tau=\Delta t$.

\section{Métodos usados para simular Mecánica Cuántica}
Se han propuesto diferentes acercamientos a la simulación de dinámica cuántica basados en 
autómatas celulares y la ecuación de Lattice Boltzmann. El primero en ser propuesto fue el 
de Sauro Succi en 1993, basado en las relaciones entre la ecuación de Dirac en su forma 
Majorana y la ecuación de Lattice Boltzmann para fluidos. Posteriormente, en 1996, David 
Meyer propone un enfoque diferente, partiendo de la relación entre la ecuación de difusión 
y la ecuación de Schrödinger para una partícula libre. Basado en esto, busca la regla de 
evolución temporal de su sistema basado en la imagen de Schrödinger, que plantea que la 
evolución temporal de un sisema esta dada por

\begin{equation}\label{eq:sch_im}
    |\psi(x,t)\rangle = \hat{U}(t, t=0)|\psi(x, t=0)\rangle
\end{equation}

Donde $\hat{U}(t, t=0)$ es el operador de evolución temporal, que es unitario, homogéneo y 
local. Por lo tanto, en una dimensión puede ser escrito como una matriz de la forma

\begin{equation*}
    \hat{U} = 
    \begin{bmatrix}
        \ddots & \\
        & a & b & c \\
        & & d & e & f \\
        & & & & & \ddots 
    \end{bmatrix}
\end{equation*}

Sin embargo, Meyer demuestra que no es posible construir un autómata celular no trivial, 
local, homogéneo y unitario en una sola dimensión \cite{meyer}por lo que relaja la 
condición de homogeneidad, manteniendo intactas las condiciones de unitareidad y localidad.
Esto convierte al operador $\hat{U}$ invariante bajo dos pasos; que quiere decir que 
$T^{-2}UT^2=U$, donde $T$ es el operador de traslación espacial.

\subsubsection{Autómata Celular}
Partiendo de esta redefinición del operador $\hat{U}$, podemos hallar la regla de 
evolución del autómata, dada por:

 \begin{equation}\label{eq:evolution}
    \left[ 
    \begin{matrix}
        \psi_{t+1}(x-1) \\
        \psi_{t+1}(x)
    \end{matrix}
    \right] = 
    \begin{bmatrix}
        i\sin\theta & \cos\theta \\
        \cos\theta & i\sin\theta
    \end{bmatrix}
    \left[ 
    \begin{matrix}
        \psi_{t}(x-1) \\
        \psi_{t}(x)
    \end{matrix}
    \right]
\end{equation}

Sin embargo, dado que el operador es invariante bajo dos pasos es necesario dividir la 
evolución en dos bloques: Colisión iniciando por pares de celdas impares seguido por un 
paso de addvección, y luego colisión de pares de celdas pares para terminar con un último 
paso de addvección. Este modelo garantiza que la evolución del sistema no conste solamente 
de interacciones entre pares de celdas que no interactúan con las demás, en lo que Meyer 
denomina el modelo de ``checkerboard''.

\subsubsection{Lattice Gas}
Extendiendo nuestro sistema podemos hallar una analogía con un automata de partículas para
difusión, en donde los contenidos de las celdas pasan a representar partículas en sí, 
en analogía con un gas. Basados de la regla de evolución para el autómata celular, podemos 
hallar la regla de evolución para este gas de partículas.

\section{Simulación}
\subsection{Autómata}
Usando la regla \ref{eq:evolution}, y tomando un espacio de Hilbert $H$ con una base 
discreta $|x\rangle$, encontramos las siguientes gráficas bonitas: 

\begin{figure*}[h]
    \centering
    \includegraphics[width=0.7\textwidth]{figs/x0.png}
    \caption{Evolución para una amplitud no nula en $x=0$}
    \label{fig:x0}
\end{figure*}

\begin{figure*}[h]
    \centering
    \includegraphics[width=0.7\textwidth]{figs/x1516.png}
    \caption{Evolución para una amplitud no nula en $x=15,16$}
    \label{fig:x15}
\end{figure*}

Donde el contenido de la celda representa la probabilidad de encontrar una partícula en 
esa posición. Vemos que el comportamiento es el esperado, con una difusión progresiva por 
un lado, y con un movimiento de la amplitud máxima.

\subsection{Lattice Gas}
A partir de la regla de evolución \ref{eq:evolution}, podemos llegar a una ecuación de la 
forma 
\begin{equation}\label{eq:QLG}
    \psi_i(x+\vec{v_i}\Delta t, t +\Delta t) = \psi_i(x,t) + (\cos\theta -1)\psi_i(x,t)
    + i\sin\theta\psi_i(x+\vec{v_i}\Delta t,t)
\end{equation}

Vemos que corresponde a la ecuación de Lattice Boltzmann, con un operador de colisión 

\begin{equation*}
    \Omega_i = -\frac{\psi_i(x,t)-\cos\theta\psi_i(x,t)
    -\sin\theta\psi_i(x+\vec{v_i}\Delta t,t)}{\tau}\Delta t
\end{equation*}

si $\Delta t = \tau$, y por lo tanto una función de equilibrio dada por

\begin{equation*}
    f_i^{eq}(x,t)=\cos\theta\psi_i(x,t)+i\sin\theta\psi_i(x+\vec{v_i}\Delta t,t)
\end{equation*}
%hay que explicar v_i
A partir de esto podemos construir un modelo de Lattice Boltzmann que permita verificar el
comportamiento de un sistema cuya solución analítica sea conocida, y validar así el uso de 
este método para el estudio de la dinámica cuántica. Para este fin tomamos un paquete 
gaussiano con un momento dado, cuya función de onda está dada por

\begin{equation*}
    \psi(x,t=0)=\frac{e^{-ikx}}{\sigma\sqrt{2\pi}}\exp{\left(-\frac{1}{2}\left(
        \frac{x-\mu}{\sigma}\right)^2\right)}
\end{equation*}

Donde la masa de la partícula está relacionada con el ángulo  mediante $m=\tan{\theta}$. 
El comportamiento esperado para un paquete gaussiano con estas características consta de 
dos partes: Por un lado, esperamos que la media siga un movimiento con velocidad 
constante, y que la desviación estándar aumente con el cuadrado del tiempo, obedeciendo 
las relaciones

\begin{equation*}
    \mu(t)=\mu_0+(km)t
\end{equation*}

\begin{equation*}
    \sigma^2(t)=\frac{t^2}{4m^2\sigma_0^2} + \sigma_0^2
\end{equation*}

Para esto tomamos un Lattice de 2048 celdas, iniciando un paquete gaussiano con media 
$\mu=1024$ y desviación estándar $\sigma^2=50$, una masa de $m=0.2$ y una velocidad de 
$v=0.1$. Los resultados para la evolución de la función de densidad de probabilidad, que 
de acuerdo a los postulados de la mecánica cuántica corresponde a $f_P(x)=|\psi(x)|^2$, y 
del cambio en la desviación estándar se encuantran en las figuras \ref{fig:gauss} y 
\ref{fig:sigma}.

\begin{figure*}[h]
    \centering
    \includegraphics[width=0.7\textwidth]{figs/latticegas.png}
    \caption{Evolución del paquete gaussiano para $t=0, 1000, 2000$ pasos de tiempo.}
    \label{fig:gauss}
\end{figure*}

\begin{figure*}[h]
    \centering
    \includegraphics[width=0.7\textwidth]{figs/new.png}
    \caption{Evolución de la desvación estándar en funcion del tiempo.}
    \label{fig:sigma}
\end{figure*}

Podemos comprobar que el comportamiento del paquete gaussiano corresponde al esperado, por 
lo que procedemos a variar la masa de la partícula. Estos resultados se encuentran 
consignados en la figura \ref{fig:mass}.

\begin{figure*}[h]
    \centering
    \includegraphics[width=0.7\textwidth]{figs/mass.png}
    \caption{Cambio de la desvación estándar en funcion del tiempo para diferentes masas.}
    \label{fig:mass}
\end{figure*}

Podemos ver que el comportamiento del paquete gaussiano corresponde al esperado 
analíticamente, validando así el modelo de Lattice Boltzmann para simular sistemas 
cuánticos.

\section{Conclusiones}
\begin{itemize}
    \item Es posible utilizar métodos computacionales para modelar sistemas cuánticos
    \item Observamos el comportamiento esperado para el autómata celular
    \item Vemos que el paquete gaussiano evoluciona de acuerdo a lo predicho
    \item La desviación estándar crece de forma cuadrática, pero presenta una oscilación, 
    debida posiblemente a errores numéricos o a las aproximaciones realizadas
    \item Vemos que la forma en la que el paquete gaussiano evoluciona depende de la masa 
    de la partícula
\end{itemize}

\nocite{*}
\printbibliography
\end{document}