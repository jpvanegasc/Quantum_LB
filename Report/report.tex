\documentclass[12pts, letterpaper, twocolumn]{article}
\usepackage[utf8]{inputenc}
\usepackage[spanish, es-tabla]{babel}
\usepackage{csquotes} %babel asked for this
\usepackage{graphicx}
\usepackage{float}
\usepackage{verbatim}
\usepackage{multicol}
\usepackage{amsmath}

\usepackage{biblatex}
\addbibresource{ref.bib}

\title{Simulación de Dinámica Cuántica usando LB}
\author{Juan B. Benavides y Juan P. Vanegas}
\date{\today}

\begin{document}
\maketitle
\section{Introducción}\label{sec:intro}
La mecanica cuántica es una de las ramas de la física más relevantes hoy en día. 
%cosas sobre cuantica
El estudio de sistemas cuánticos utilizando herramientas computacionales introduce una 
visión totalmente nueva sobre estos, abriendo todo un nuevo paradigma de estudio. Por un 
lado, permite ver la evolución de sistemas más complejos, sin necesidad de solucionar 
analíticamente las ecuaciones que los describen. Por otro lado, es posible implementar 
nuevos algoritmos que incluyan correción de efectos cuánticos.

Dentro de los métodos computacionales para simular sistemas físicos, podemos destacar dos 
que a primera vista pueden ser apropiados para la simulación de mécanica cuántica: 
Autómatas Celulares y Lattice Boltzmann. El motivo por el cuál destacan se debe a la 
naturaleza de su estructura: No se basan en comportamientos deterministas de particulas, 
sino en la evolución de un sistema en la que cada punto del espacio interactúa de acuerdo 
con alguna regla con sus vecinos, similar a la evolución de los sistemas cuánticos.

\subsection{Autómata Celular}
El método de autómatas celulares, desarrollado en (cuando? por quien?), consta en un 
arreglo de celdas, y un conjunto de reglas de evolución, que relacionan el comportamiento
de cada celda basado en sus vecinas. Este método resulta especialmente útil para modelar 
sistemas difusivos, o %bla bla bla

\subsection{Lattice Boltzmann}
El método de Lattice Boltzmann nace de la dinámica de fluidos y de la ecuación de 
Boltzmann, que, al ser discretizada, permite obtener una ecuación que depende 
exclusivamente de 
Encontramos entonces una función de parámetros discretos $f$, que se relaciona con las 
variables macroscópicas del sistema mediante las relaciones 

\begin{equation*}
    \rho(\textbf{x}, t) = \sum_i f_i(\textbf{x}, t)
\end{equation*}

\begin{equation*}
    \rho\textbf{u}(\textbf{x}, t) = \sum_i \vec{v_i} f_i(\textbf{x}, t)
\end{equation*}

Donde el subíndice $i$ pertenece a un conjunto de direcciones discreto, en las cuales se 
puede propagar la información.


\begin{equation}\label{eq:LB}
    f_i(\textbf{x}+\vec{c_i} \Delta t, t+\Delta t) = f_i(\textbf{x}, t) + \Omega_i(\textbf{x}, t)
\end{equation}

Donde el operador de colisión $\Omega_i$ más simple es el de Bhatnagar-Gross-Krook (BGK), 
definido como 

\begin{equation*}
    \Omega_i = -\frac{f_i-f_i^{eq}}{\tau}\Delta t
\end{equation*}

Donde la función de equilibrio depende del sistema que se quiera modelar (fluidos, difusión, etc.).


\section{Metodos para simular MC}
Se han propuesto diferentes acercamientos a la simulación de dinámica cuántica basados en 
autómatas celulares y la ecuación de Lattice Boltzmann. El primero en ser propuesto fue el 
de Sauro Succi en 1993, basado en las relaciones entre la ecuación de Dirac en su forma 
Majorana y la ecuación de Lattice Boltzmann para fluidos. Posteriormente, en 1996, David 
Meyer propone un enfoque diferente, partiendo de la relación entre la ecuación de difusión 
y la ecuación de Schrödinger para una partícula libre. Basado en esto, busca la regla de 
evolución temporal de su sistema basado en la imagen de Schrödinger, que plantea que la 
evolución temporal de un sisema esta dada por

\begin{equation}\label{eq:sch_im}
    |\psi(x,t)\rangle = \hat{U}(t, t=0)|\psi(x, t=0)\rangle
\end{equation}

Donde $\hat{U}(t, t=0)$ es el operador de evolución temporal, que es unitario, homogéneo y 
local. Por lo tanto, en una dimensión puede ser escrito como una matriz de la forma

\begin{equation*}
    \hat{U} = 
    \begin{bmatrix}
        \ddots & \\
        & a & b & c \\
        & & d & e & f \\
        & & & & & \ddots 
    \end{bmatrix}
\end{equation*}

Sin embargo, Meyer demuestra que no es posible construir un autómata celular no trivial, 
local, homogéneo y unitario en una sola dimensión \cite{meyer}por lo que relaja la 
condición de homogeneidad, manteniendo intactas las condiciones de unitareidad y localidad.
Esto convierte al operador $\hat{U}$ invariante bajo dos pasos; que quiere decir que 
$T^{-2}UT^2=U$, donde $T$ es el operador de traslación espacial.

\subsubsection{Automata Celular}
 Para nuestro caso tomamos una matriz unidimensional, y usando la ecuación de Schrödinger 
 encontramos la regla de evolución \cite{meyer}

 \begin{equation}\label{eq:evolution}
    \left[ 
    \begin{matrix}
        \psi_{t+1}(x-1) \\
        \psi_{t+1}(x)
    \end{matrix}
    \right] = 
    \begin{bmatrix}
        i\sin\theta & \cos\theta \\
        \cos\theta & i\sin\theta
    \end{bmatrix}
    \left[ 
    \begin{matrix}
        \psi_{t}(x-1) \\
        \psi_{t}(x)
    \end{matrix}
    \right]
\end{equation}
\subsubsection{Lattice Gas}
Extendiendo nuestro sistema podemos hallar una analogía con un automata celular de difusión



\section{Simulación}
\subsection{Autómata}

\begin{figure*}[h]
    \centering
    \includegraphics[width=\textwidth]{figs/x0.png}
    \caption{amplitud no nula en x=0}
    \label{fig:x0}
\end{figure*}

\begin{figure*}[h]
    \centering
    \includegraphics[width=\textwidth]{figs/x1516.png}
    \caption{amplitud no nula en x=15, 16}
    \label{fig:x15}
\end{figure*}


\subsection{Lattice Gas}

A partir de la regla de evolucion \ref{eq:evolution}, podemos llegar a una ecuación de la 
forma 
\begin{equation}\label{eq:QLG}
    \psi_i(x+\vec{v_i}\Delta t, t +\Delta t) = \psi_i(x,t) + (\cos\theta -1)\psi_i(x,t)
    + i\sin\theta\psi_i(x+\vec{v_i}\Delta t,t)
\end{equation}

Vemos que corresponde a la ecuación de Lattice Boltzmann, con un operador de colisión 

\begin{equation*}
    \Omega_i = -\frac{\psi_i(x,t)-\cos\theta\psi_i(x,t)
    -\sin\theta\psi_i(x+\vec{v_i}\Delta t,t)}{\tau}\Delta t
\end{equation*}

Si $\Delta t = \tau$, y por lo tanto una función de equilibrio dada por

\begin{equation*}
    f_i^{eq}(x,t)=\cos\theta\psi_i(x,t)+i\sin\theta\psi_i(x+\vec{v_i}\Delta t,t)
\end{equation*}
%hay que explicar v_i
A partir de esto podemos construir un modelo de Lattice Boltzmann que permita verificar el
comportamiento de un sistema cuya solución analítica sea conocida, y validar así el uso de 
este método para el estudio de la dinámica cuántica. Para este fin tomamos un paquete 
gaussiano con un momento dado, cuya función de onda está dada por

\begin{equation*}
    \psi(x,t=0)=\frac{e^{-ikx}}{\sigma\sqrt{2\pi}}\exp{\left(-\frac{1}{2}\left(
        \frac{x-\mu}{\sigma}\right)^2\right)}
\end{equation*}

Esperamos que 

\begin{figure*}[h]
    \centering
    \includegraphics[width=\textwidth]{figs/latticegas.png}
    \caption{evlucion paquete gaussiano}
    \label{fig:gauss}
\end{figure*}

\begin{figure*}[h]
    \centering
    \includegraphics[width=\textwidth]{figs/new.png}
    \caption{desvación estándar en funcion del tiempo}
    \label{fig:sigma}
\end{figure*}

\begin{figure*}[h]
    \centering
    \includegraphics[width=\textwidth]{figs/mass.png}
    \caption{desvación estándar en funcion del tiempo para diferentes masas}
    \label{fig:mass}
\end{figure*}


\section{Resultados}
\section{Conclusión}
\nocite{*}
\printbibliography
\end{document}